\chapter{PENUTUP}
\section{Kesimpulan}
Kampus Merdeka memberikan kesempatan bagi mahasiswa untuk menjelajahi keilmuan luar program studi melalui program studi independen. Melalui Bangkit Academy, program Kampus Merdeka berhasil mendidik ribuan mahasiswa di bidang IT, mempersiapkan mahasiswa untuk terjun di dunia kerja, dan menghasilkan lulusan yang terdidik baik dari ilmu eksakta maupun \textit{soft skills}.

\textit{Partner} dari program Bangkit Academy juga berhasil memperkenalkan mahasiswa dengan solusi-solusi IT yang dipergunakan oleh perusahaan-perusahaan besar di dunia, seperti Google Cloud Platform. Bangkit juga dapat mempergunakan berbagai platform sesuai dengan peran dan tujuannya, seperti Dicoding untuk pemahaman dasar, Coursera untuk pemahaman mendetail, dan Qwiklabs untuk pembelajaran \textit{hands-on} dengan Google Cloud Platform. Proyek Capstone juga mendidik mahasiswa untuk bekerja dalam tim dan dalam tekanan \textit{deadline}.
\section{Saran}
Untuk peningkatan pencapaian di masa depan, terdapat beberapa saran yang bisa dipertimbangkan untuk kegiatan ini:
\subsection{Untuk pihak Bangkit}
\begin{enumerate}
	\item Penambahan jumlah ILT dan pemotongan durasi waktu ILT untuk menjaga fokus mahasiswa.
	\item Pembuatan sebuah website portal yang terpusat untuk urusan nilai, berita, dan komunikasi kelas.
\end{enumerate}
\subsection{Untuk pihak kampus}
\begin{enumerate}
	\item Meningkatkan pemasaran program MBKM di iGracias untuk meningkatkan jumlah peminat dari Telkom University.
\end{enumerate}
\subsection{Untuk mahasiswa}
\begin{enumerate}
	\item Menjaga fokus dan motivasi terhadap program.
	\item Rajin membuat logbook Kampus Merdeka.
	\item Memanfaatkan kesempatan yang diberikan Bangkit dengan sebaik-baiknya.
\end{enumerate}