\chapter{PENDAHULUAN}
\section{Latar Belakang}
Bangkit Academy merupakan sebuah program studi independen oleh Dicoding Indonesia yang bekerja sama dengan Google, GoTo, dan Traveloka. Merupakan bagian dari program Merdeka Belajar Kampus Merdeka oleh Kementerian Pendidikan, Kebudayaan, Riset, dan Teknologi (Kemendikbudristek), peserta Bangkit Academy merupakan mahasiswa aktif semester 5-8 dari seluruh Indonesia dari berbagai program studi.

Sebagai bagian dari program Kampus Merdeka, peserta Bangkit Academy dituntut bertanggung jawab atas pembelajaran modulnya masing-masing, dimana proses belajar-mengajar dilaksanakan secara daring dan ekstra-kampus. Pada tahun 2022, terdapat tiga \textit{learning path} dari program Bangkit Academy, yakni: \textit{Cloud Computing, Machine Learning,} dan \textit{Android Development.}

Proses kegiatan Bangkit Academy terdiri dari beberapa tipe kegiatan: pembelajaran secara sinkron tatap muka, pembelajaran secara asinkron, \textit{capstone project}, dan sertifikasi. Pada Bangkit, peserta juga diajarkan cara berkomunikasi secara profesional, cara memulai karir dan mempersiapkan diri untuk terjun ke dunia kerja, keahlian bisnis, berbahasa Inggris, serta melatih \textit{skill} bekerja sama dengan peserta lainnya.

\section{Lingkup Penugasan}
Lingkup pelaksanaan kegiatan Bangkit 2022 adalah sebagai berikut:
\begin{itemize}
	\item Tanggal: 14 Februari 2022 - 27 Juli 2022
	\item Tempat: Daring
	\item Hari kegiatan: Senin - Jumat
	\item Waktu kegiatan: 08.00 - 18.00
\end{itemize}

\section{Target Pemecahan Masalah}
Berikut adalah target pemecahan masalah selama pelaksanaan Bangkit 2022, \textit{Cloud Computing Path}:
\begin{enumerate}
	\item Memahami ilmu dasar \textit{web development} dengan bahasa HTML, CSS, JavaScript.
	\item Memahami konsep cara kerja sebuah \textit{website} dan \textit{web-based application} dari \textit{frontend} hingga \textit{backend}.
	\item Memahami arsitektur jaringan Google Cloud Platform
	\item Memahami cara berkomunikasi secara profesional, membuat \textit{personal branding}, bisnis berbasis IT, serta kewirausahaan.
\end{enumerate}

\section{Metode Pelaksanaan Tugas}
Dalam pelaksanaan Bangkit 2022, metode pelaksanaan tugas dibedakan berdasarkan \textit{learning path}. Penulis mengambil \textit{learning path Cloud Computing} menggunakan metode \textit{self-learning,} dan topik pembelajaran yang didapatkan meliputi:
\begin{enumerate}
	\item Bangkit \textit{Cloud Computing}:
	\begin{enumerate}
		\item Dasar Pemrograman Web
		\item Dasar Pemrograman JavaScript
		\item Pembuatan Aplikasi Back-End untuk Pemula dengan Google Cloud
		\item Google Cloud Computing Foundations
		\item Architecting with Google Compute Engine
		\item Associate Cloud Engineer Certification
	\end{enumerate}
	\item Bangkit \textit{Soft Skills}
	\begin{enumerate}
		\item \textit{Time Management}
		\item \textit{Professional Branding and Interview Communications}
		\item \textit{Critical Thinking}
		\item \textit{Adaptability}
		\item \textit{Idea Generation and MVP Planning}
		\item \textit{Startup Valuation and Investment Pitch}
		\item \textit{Professional Communications}
	\end{enumerate}
	\item Bangkit \textit{Capstone Project}
\end{enumerate} 

\section{Rencana dan Penjadwalan}
Bangkit 2022 dimulai pada tanggal 14 Februari 2022 dan berakhir pada tanggal 27 Juli 2022, terhitung selama satu semester perkuliahan. Kegiatan dilaksanakan secara sepenuhnya daring, serta dapat dikonversikan hingga sebanyak 20 SKS.

Peserta \textit{Cloud Computing Learning Path} menggunakan beragam \textit{platform} belajar, yakni: Google Classroom untuk penjadwalan dan penugasan, Google Meet untuk pembelajaran secara sinkron, Dicoding dan Coursera untuk pembelajaran asinkron, serta Qwiklabs untuk pembelajaran \textit{hands-on} dengan Google Cloud Platform. Jadwal kegiatan berbeda-beda tergantung grup peserta dan \textit{instructor}.

\section{Sistematika Laporan}
Sistematika penulisan pada laporan magang program Bangkit 2022 ini menggunakan sistematika yang telah ditentukan pada "Buku Panduan Kerja Praktek 2022".
\begin{description}
	\item[BAB I PENDAHULUAN] Bab ini membahas latar belakang, lingkup penugasan kerja praktek, target pemecahan masalah, rencana dan penjadwalan kerja, serta ringkasan sistematika laporan.
	\item[BAB II PROFIL INSTITUSI KERJA PRAKTEK] Bab ini berisi profil institusi, struktur organisasi institusi, serta lokasi dan unit pelaksanaan kerja.
	\item[BAB III KEGIATAN KERJA PRAKTEK DAN PEMBAHASAN KRITIS] Bab ini berisi tentang kegiatan mahasiswa dalam kerja praktek, pemaparan data yang didapat dari kerja praktek, serta analisis kritis mahasiswa atas data tersebut.
	\item[BAB IV KESIMPULAN DAN SARAN] Bab ini berisi kesimpulan mengenai kegiatan kerja praktek, serta saran untuk instansi maupun universitas.
\end{description}