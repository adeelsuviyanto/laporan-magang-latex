%
% Konfigurasi LaTeX Laporan Magang
%
% Muhammad Adeel Mahdi Suviyanto (1102183191)
% S1 Teknik Elektro
% Fakultas Teknik Elektro
% Telkom University
%
%
%

%Data buku
%Judul laporan
\newcommand{\judul}{\textit{BrainDiction}: Aplikasi Android Pendeteksi Tumor Otak Berbasis Machine Learning di Google Cloud Platform}
%Judul laporan, kapital
\newcommand{\JUDUL}{\textit{BRAINDICTION}: APLIKASI ANDROID PENDETEKSI TUMOR OTAK BERBASIS MACHINE LEARNING DI GOOGLE CLOUD PLATFORM}

%Tahun publikasi
\newcommand{\tahun}{2022}
%Tanggal pengesahan
\newcommand{\tanggalpengesahan}{x Juli 2022}

%Data penulis
%Nama penulis
\newcommand{\penulis}{Muhammad Adeel Mahdi Suviyanto}
%Nama penulis, kapital
\newcommand{\PENULIS}{MUHAMMAD ADEEL MAHDI SUVIYANTO}
%NIM
\newcommand{\nim}{1102183191}
%Email penulis
\newcommand{\email}{adeelsuviyanto@gmail.com}
%Program Studi
\newcommand{\prodi}{S1 Teknik Elektro}
%Gelar yang akan diperoleh
\newcommand{\gelar}{Sarjana Teknik}
%Program kuliah
\newcommand{\program}{Teknik Elektro Universitas Telkom}
%Fakultas
\newcommand{\fakultas}{Fakultas Teknik Elektro}
%Fakultas, kapital
\newcommand{\FAKULTAS}{FAKULTAS TEKNIK ELEKTRO}
%Universitas
\newcommand{\universitas}{Universitas Telkom}

%Data Pembimbing
%Pembimbing 1
\newcommand{\pembimbingsatu}{Dr. Eng. Willy Anugrah Cahyadi, S.T, M.T}
\newcommand{\pembimbingdua}{Angga Rusdinar, S.T., M.T., Ph.D.}

%
% Document packages!
%
% Atur sesuai kebutuhan, masih berbasis pada format Proposal Tugas Akhir
%
\usepackage[indonesian]{babel}
\usepackage{amsmath} %Penulisan notasi matematika
\usepackage{amsfonts}
\usepackage{amssymb}
\usepackage{graphicx} %Memasukkan gambar ke laporan
\usepackage{fontenc}
\usepackage{pslatex} %Times New Roman
\usepackage{hyperref}
\usepackage[a4paper, lmargin=4cm, rmargin=3cm, tmargin=3cm, bmargin=3cm]{geometry}
\usepackage[backend=biber, style=ieee, language=english]{biblatex}
\usepackage{csquotes}
\usepackage{fancyhdr} %Header and footer
\usepackage{longtable}
\usepackage{caption} %Caption gambar
\usepackage{setspace}
	\onehalfspacing %Set line spacing ke 1.5
\usepackage[ConnyRevised]{fncychap}
\usepackage{titlesec}
\usepackage{tocloft}
\usepackage{enumitem}
\usepackage{indentfirst}
\usepackage{float}
%
% Chapter style setup
%
\titleformat{\chapter}[display]{\bfseries\large\centering}{BAB \thechapter}{1pt}{}[]
\titlespacing*{\chapter}{0pt}{-50pt}{1.5pt}
%setup section style
\titleformat{\section}[hang]{\bfseries\normalsize}{\thesection \ }{1pt}{}[]
\titlespacing*{\section}{0pt}{0pt}{1.5pt}
\titleformat{\subsection}[hang]{\bfseries\normalsize}{\thesubsection \ }{1pt}{}[]
\titleformat{\subsubsection}[hang]{\bfseries\normalsize}{\thesubsubsection \ }{1pt}{}[]

%set biblatex file
\addbibresource{./src/ta.bib}

%disable numbering for front matter
\newenvironment{unnumbered}
{\global\chardef\keeplevel=\value{secnumdepth}%
	\setcounter{secnumdepth}{-1}}
{\setcounter{secnumdepth}{\keeplevel}}

%dotted line for TOC
\renewcommand{\cftpartleader}{\cftdotfill{\cftdotsep}} % for parts
\renewcommand{\cftchapleader}{\cftdotfill{\cftdotsep}} % for chapters